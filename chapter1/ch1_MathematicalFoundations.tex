\chapter{Mathematical Foundations}

The objective of this chapter is to present some basic concepts that underlie much of the remaining material in this book.

\section{Basic Notation}

The reader may be familiar with some if not all of the following notation, but it is presented here for concreteness and review.

\begin{itemize}
\item $\forall$ is read as ``for all"  
\item $\exists$ is read as ``exists"    
\item $\subset$ is read as ``subset of"  
\item $\in$ is read as ``in"  
\item $\mathbb{Z}$ is the set of integers: $\ldots, -2,-1,0,1,2, 3, \ldots$
\item $\mathbb{N}$ is the set of natural numbers: $0,1,2, 3, \ldots$ (note: Sometimes these start with 0, and sometimes with 1.  Usually doesn't matter to the argument in question, but the reader should be aware.)
\item $\mathbb{Q}$ is the set of rational numbers.  These are numbers of the form $\frac{p}{q}$ where $p,q$ are integers, and $q \ne 0$
\item $\mathbb{R}$ is the set of real numbers.  These will later be defined precisely in terms of \textit{Dedekind Cuts},\index{Dedekind Cuts} but for now we'll just use the informal notion of all the points on a coordinate line.  However, hopefully the reader can see this is not good enough for precise work.
\item We use curly braces to indicate sets in general.  For example, $\mathbb{Z}=\left\{\ldots, -2,-1,0,1,2,3 \ldots\right\}$ defines the set of integers using the curly braces and examples.  It is hoped the reader is able to abstract that the dots to the left imply the list has no smallest negative value and the dots to the right imply that there is no largest positive value.  
\item Set builder notation:\index{Set builder notation}  There are other ways to specify the set within the braces.  For example, we can indicate the property that each member of the set has:  $S = \left\{p | p=2m+1, m \in \mathbb{Z} \right\}.$  This is read as the set $S$ consists of those values $p$ of the form $2m+1$ where $m$ is an integer.  In other words, $S$ the set of odd integers.  The vertical bar in the braces is read as ``such that" and the comma is read as ``and.''
\end{itemize}

\textit {Exercise}

Write the definition of the set of rational numbers using set builder notation.

\textit{Solution}

$\mathbb{Q}=\left\{r | r = \frac{p}{q}, p,q \in \mathbb{Z}, q \ne 0 \right\}$

\section{Mathematical Logic}

It should be remarked that a good portion of this section (as well as some others) was strongly inspired by the YouTube video series ``Math Major Basics" by MathDoctorBob (https://www.youtube.com/playlist?list=PLF2DF6C3C8015DF5F).  We cover some key aspects here.  It is very helpful to understand \textit{Truth Tables} in this context.  For those who have some experience with digital logic or computer programming, much of this will be quite familiar.

We have the following logical relations, with symbols, defined by the truth tables shown below:  Negation ($\neg$), Conjunction/And ($\land$ - note that this symbol looks like the outline of the letter ``A", to help you remember ``And"), Disjunction/or ($\lor$), Implication ($\Longrightarrow$), If and Only If ($\Longleftrightarrow$):

\textbf{Negation:}
\begin{displaymath}
\begin{array}{|c|c|}
A & \neg A \\
\hline
T & F \\
F & T \\
\end{array}
\end{displaymath}

\textbf{Conjunction/And:}
\begin{displaymath}
\begin{array}{|c|c|c|}
A & \land & B  \\
\hline
T & T & T  \\
T & F & F  \\
F & F & T  \\
F & F & F  \\
\end{array}
\end{displaymath}

\textbf{Disjunction/Or:}
\begin{displaymath}
\begin{array}{|c|c|c|}
A & \lor & B  \\
\hline
T & T & T  \\
T & T & F  \\
F & T & T  \\
F & F & F  \\
\end{array}
\end{displaymath}

\textbf{Implication:}
\begin{displaymath}
\begin{array}{|c|c|c|}
A & \Longrightarrow & B \\
\hline
T & T & T  \\
T & F & F  \\
F & T & T  \\
F & T & F  \\
\end{array}
\end{displaymath}

\textbf{If and only if (Iff):}  
\begin{displaymath}
\begin{array}{|c|c|c|}
A & \Longleftrightarrow & B \\
\hline
T & T & T  \\
T & F & F  \\
F & F & T  \\
F & T & F  \\
\end{array}
\end{displaymath}

In the above, $A$ and $B$ are statements which can be either true (T) or false (F).  The result of the logical operation is shown in the column under the particular symbol.  For example, if A is true, and B is true, then $A \land B$ is true.  However, either either $A$ or $B$ is false (or both) then $A \land B$ is false.  It is important to refer back to the truth table definitions of these logical operations to avoid confusion.\\  

Sometimes, implication is called ``if-then" as in ``If A then B."  However, note that it is possible for A to be false and B to be true, and the implication $A \Longrightarrow B$ to be true.  This is somewhat at odds with how this if-then might be expressed in common usage: How could something false imply something true?  This seems confusing.  Therefore, always refer to the truth table for clarity on this point.\\

Here is an illustration regarding implication ($\Longrightarrow$).

Let $A=(a > 5), B=(a > 3)$ where $a$ is some integer.  Now consider the following cases: \\
If $a=6, (6 > 5)=\textit{T}, (6 > 3)=\textit{T}$ \\  
If $a=4, (4 > 5)=\textit{F}, (4 > 3)=\textit{T}$  \\
If $a=2, (2 > 5)=\textit{F}, (2 > 3)=\textit{F}$ \\

If we compare this to the truth table for implication, we see that implication is always true.  Note that with this particular definition of the statements $A$ and $B$ we cannot have a case where $A$ is true and $B$ is false. \\ 

\textit{Exercise}
Show $A  \Longleftrightarrow  B$ is equivalent to $\left(A \Longrightarrow B \right) \land \left(B \Longrightarrow A \right).$ 
\\
\\
\textit{Solution} \\
\begin{displaymath}
\begin{array}{|c|c|c|c|c|c|c|}
(A & \Longrightarrow & B)  & \land & (B & \Longrightarrow & A ) \\
\hline
T & T & T & T & T & T & T  \\
T & F & F & F & F & T & T  \\
F & T & T & F & T & F & F  \\
F & T & F & T & F & T & F  \\
\end{array}
\end{displaymath}

We now consider some important terminology:

\begin{itemize}
\item \textit{Tautology}:  Any statement that is true for all values of its components is a tautology.

\item \textit{Contradiction}:  Any statement that is false for all values of its components is a contradiction.  The negation of a contradiction is a tautology and the negation of a tautology is a contradiction.

\item \textit{Logically Equivalent}:  When $ A \Longleftrightarrow B $ is a tautology.  This means $A$ and $B$ have the same truth table.  The notation for this is $A \equiv B$.

\end{itemize}

We illustrate logical equivalence by considering ``De Morgan's Laws" \index{De Morgan's Law's} for disjunction (or) and conjunction (and).  De Morgan's Law's are useful for manipulating logical expressions: \\

$\neg\left(A \lor B \right) \equiv \left(\neg A\right) \land \left(\neg B \right)$

$\neg\left(A \land B \right) \equiv \left(\neg A\right) \lor \left(\neg B \right)$ \\

In words, the first law says that the negation $(\neg)$ of a disjunction ($A \lor B)$ is logically equivalent to the conjunction of the negations of the individual statements $A$ and $B$.  The second law is similar, except we exchange conjunction for disjunction and vice-versa.\\

\textit{Exercise}
Verify the De Morgan law for disjunction by showing the logical equivalence of the left and right hand sides using a truth table:\\
\textit{Solution}\\


\begin{displaymath}
\begin{array}{| c | c | c | c | c | c | c | c | c | c |}
\neg & (A & \lor  & B) & \Longleftrightarrow & (\neg & A) & \land & (\neg & B) \\
\hline
F & T & T & T & T & F & T & F & F & T \\
F & T & T & F & T & F & T & F & T & F \\
F & F & T & T & T & T & F & F & F & T \\
T & F & F & F & T & T & F & T & T & F \\
\end{array}
\end{displaymath}

We note that for any value of statements $A$ and $B$, the values under the column for $\Longleftrightarrow$ are all true, indicating that both expressions $\neg\left(A \lor B \right)$, $\left(\neg A\right) \land \left(\neg B \right)$ have the same truth tables, as required.\\

Another useful logical equivalence is the following ``contrapositive'' form of implication, which we explore in the next exercise:

\textit{Exercise}
Show that $\left(A \Longrightarrow B \right) \Longleftrightarrow \left(\neg B \Longrightarrow \neg A\right)$ is a tautology, i.e. this means that $A \Longrightarrow B $ is logically equivalent (LE) to $\neg B \Longrightarrow \neg A$.

\textit{Solution}\\

\begin{displaymath}
\begin{array}{| c | c | c | c | c | c | c | c | c |}
(A & \Longrightarrow & B)& \Longleftrightarrow & (\neg & B & \Longrightarrow & \neg & A) \\
\hline
T & T & T & T & F & T & T & F &T \\
T & F & F & T & T & F & F & F & T \\
F & T & T & T & F & T & T & T & F \\
F & T & F & T & T & F & T & T & F \\
\end{array}
\end{displaymath}


To show a situation where expressions are not logicaly equivalent, the following exercise may be considered:

\textit{Exercise} 
Show $A \Longrightarrow B$ is not logically equivalent to $B \Longrightarrow A$

\textit{Solution}\\

\begin{displaymath}
\begin{array}{| c | c | c | c | c | c | c |}
(A & \Longrightarrow & B) & \Longleftrightarrow & (B & \Longrightarrow & A) \\
\hline
T & T & T & T & T & T & T \\
T & F & F & F & F & T & T \\
F & T & T & F & T & F & F \\
F & T & F & T & F & T & F \\
\end{array}
\end{displaymath}

Notice the middle column $\Longleftrightarrow$ has false entries, so we don't have logical equivalency.\\

Another item of terminology is \textit{Logical Implication}.  Logical implication is when the expression $A \Longrightarrow B$ is a tautology.  If the statements $A$ and $B$ are such that we only get true statements from $A \Longrightarrow B$, then $A$ logically implies \textbf{(LI)} $B$.  This was the situation with the earlier illustration where we had defined statements $A$ and $B$ such that $A=(a > 5), B=(a > 3)$ where $a$ is some integer.

If $A$ is logically equivalent (LE) to $B$, then $A$ LI $B$ and $B$ LI $A$.

\textit{Exercise}:  Show $A\land \left(A \Longrightarrow B\right)$ logically implies $B$.  This is called ``Modus Ponens.''  

\textit{Solution}\\

\begin{displaymath}
\begin{array}{| c | c | c | c | c | c | c |}
 (A & \land & (A & \Longrightarrow & B)) & \Longrightarrow & B \\
\hline
T & T & T & T & T & T & T \\
T & F & T & F & F & T & F \\
F & F & F & T & T & T & T \\
F & F & F & T & F & T & F \\
\end{array}
\end{displaymath}


\section{Zermelo-Fraenkel Axiomatic Set Theory}

Set theory is at the foundation of mathematics.  The reader should have some awareness of the axioms of set theory, although in practice it seems that one rarely sees a mathematical proof that explicitly cites these axioms.  These axioms are referred to as the Zermelo-Frankel Axioms.

\subsection{The Zermelo-Fraenkel Axioms}

The following so-called Zermelo-Fraenkel Axioms of set theory are presented to help provide the reader a more comprehensive picture of mathematics foundations.  This section is inspired by the set of YouTube lectures by Richard Borcherds \href{https://www.youtube.com/playlist?list=PL8yHsr3EFj52EKVgPi-p50fRP2_SbG2oi}{(ZF Axioms)}\\
\verb+(https://www.youtube.com/playlist?list=PL8yHsr3EFj52EKVgPi-p50fRP2_SbG2oi)+\\ and also the book \textit{Axiomatic Set Theory} by Paul Bernays.  The definitions are adapted from Bernays.\\

We note that in the axioms below, the notion of \textit{set} is undefined, as is the inclusion relation, $\in$.  Furthermore, all elements or members of sets are sets themselves.\\

\textit{Axiom of Extensionality}

If $s \subseteq t$ and $t \subseteq s$, then $s=t$ where $s,t$ are sets and $\subseteq$ means \textit{subset}.\\  

We define subset by the following:  If $s$ and $t$ are sets such that, for all $x$, $x \in s$ implies $x \in t$, $s$ is called a subset of $t$, where $x$ is a set element or member.  Occasionally, we might use the notation $s \subset t$ to indicate a \textit{proper subset} whereby there is an element in $t$ that is not in $s$. \\

To recap:  The axiom states what it means for two sets to be equal - namely they contain the same elements.\\

\textit{Axiom of Foundation}

Every non-empty set $s$ contains an element $t$ such that $s$ and $t$ have no common element.\\

This axiom prevents an element of a set having an element which appears in the original set.  For example, the set $s=\{s\}$ is not allowed.

\textit{Axiom of Pairing}

For any two different sets $a$ and $b$, the pair $\{a,b\}$, or $\{b,a\}$, exists.\\

\textit{Axiom of Union}

For any set $s$ which contains at least two elements, there exists the set whose elements are the elements of the elements of $s.$ \\ 

This set is called the union of the elements of $s$ and is denoted by $\cup s.$  \\

\textit{Axiom of Infinity}

There exists at least one set $W$ with the properties:

\begin{itemize}
\item a) $\emptyset \in W$, where $\emptyset$ refers to the ``empty set." - the set with no elements.
\item b) if $x \in W$, then $\{x\} \in W.$
\end{itemize}

The interpretation of the above is that a set exists with a non-ending sequence of elements.  For example, if we think of the infinite sequence of non-negative integers being encoded as follows:  $0 = \emptyset, 1=\{ 0 \}, 2=\{1\}, \ldots$, then we can have the corresponding infinite set $S=\{\emptyset, \{\emptyset\}, \{\{\emptyset\}\}, \ldots \}$ which is justified by the axiom.\\

\textit{Axiom of Power-set}

For any set $s$, there exists the set whose elements are all subsets of $s$.\\

To take an example, consider the set $s = \{1,2,3 \}$.  Then the power-set is $P=\{\{1\}, \{2\}, \{3\}, \{1,2\}, \{1,3\}, \{2,3\}, \{1,2,3\}, \emptyset \}$.  The number of elements, also called the \textit{cardinality}\index{Cardinality}, in this case is given by $2^{N}$, where $N$ is the number of elements in the original set $s$.  Thus, we have $2^{N=3}=8$.  We note that the empty set, $\emptyset$, is considered a subset of every set.\\

\textit{Axiom of Subsets}  (Note: Borcherds refers to this as ``separation")

For any set $s$ and any predicate $P$ which is meaningful for all elements of $s$, there exists the set $y$ that contains just those elements $x$ of $s$ which satisfy the predicate $P$. \\

For example, let $s=\{1,2,3,4,5\}$ and the predicate $P\left(x\right)$ be true whenever $x$ is even.  Then, we have the set $y = \{x \in s | P (x) \textit{true} \} = \{2,4\}$.\\

\textit{Axiom of Replacement}

\textit{Axiom of Choice}

\subsection{Set Operations}

In this section, we present some set operations that are useful in much of the remainder of this book.  The reader may be familiar with some if not most of the material, so hopefully this review will remind and reinforce these concepts.  Concepts and proofs in mathematics are very frequently expressed in terms of set language to be precise, so utility with set operations is important.  For example, in later sections, we will discuss algebraic structures such as ``groups'' which will be defined in terms of sets with certain properties.


\section{Peano Axioms of Arithmetic}

\section{Godel Theorems}