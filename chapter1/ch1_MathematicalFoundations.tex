\chapter{Mathematical Foundations}

The objective of this chapter is to present some basic concepts that underlie much of the remaining material in this book.

\section{Basic Notation}

The reader may be familiar with some if not all of the following notation, but it is presented here for concreteness and review.

\begin{itemize}
\item $\forall$ is read as ``for all"  
\item $\exists$ is read as ``exists"    
\item $\subset$ is read as ``subset of"  
\item $\in$ is read as ``in"  
\item $\mathbb{Z}$ is the set of integers: $\ldots, -2,-1,0,1,2, 3, \ldots$
\item $\mathbb{N}$ is the set of natural numbers: $0,1,2, 3, \ldots$ (note: Sometimes these start with 0, and sometimes with 1.  Usually doesn't matter to the argument in question, but the reader should be aware.)
\item $\mathbb{Q}$ is the set of rational numbers.  These are numbers of the form $\frac{p}{q}$ where $p,q$ are integers, and $q \ne 0$
\item $\mathbb{R}$ is the set of real numbers.  These will later be defined precisely in terms of \textit{Dedekind Cuts},\index{Dedekind Cuts} but for now we'll just use the informal notion of all the points on a coordinate line.  However, hopefully the reader can see this is not good enough for precise work.
\item We use curly braces to indicate sets in general.  For example, $\mathbb{Z}=\left\{\ldots, -2,-1,0,1,2,3 \ldots\right\}$ defines the set of integers using the curly braces and examples.  It is hoped the reader is able to abstract that the dots to the left imply the list has no smallest negative value and the dots to the right imply that there is no largest positive value.  
\item Set builder notation:\index{Set builder notation}  There are other ways to specify the set within the braces.  For example, we can indicate the property that each member of the set has:  $S = \left\{p | p=2m+1, m \in \mathbb{Z} \right\}.$  This is read as the set $S$ consists of those values $p$ of the form $2m+1$ where $m$ is an integer.  In other words, $S$ the set of odd integers.  The vertical bar in the braces is read as ``such that" and the comma is read as ``and.''
\end{itemize}

\textit {Exercise}

Write the definition of the set of rational numbers using set builder notation.

\textit{Solution}

$\mathbb{Q}=\left\{r | r = \frac{p}{q}, p,q \in \mathbb{Z}, q \ne 0 \right\}$

\section{Mathematical Logic}

It should be remarked that a good portion of this section (as well as some others) was strongly inspired by the YouTube video series ``Math Major Basics" by MathDoctorBob (https://www.youtube.com/playlist?list=PLF2DF6C3C8015DF5F).  We cover some key aspects here.  It is very helpful to understand \textit{Truth Tables} in this context.  For those who have some experience with digital logic or computer programming, much of this will be quite familiar.

We have the following logical relations, with symbols, defined by the truth tables shown below:  Negation ($\neg$), Conjunction/And ($\land$ - note that this symbol looks like the outline of the letter ``A", to help you remember ``And"), Disjunction/or ($\lor$), Implication ($\Longrightarrow$), If and Only If ($\Longleftrightarrow$):

\textbf{Negation:}
\begin{displaymath}
\begin{array}{|c|c|}
A & \neg A \\
\hline
T & F \\
F & T \\
\end{array}
\end{displaymath}

\textbf{Conjunction/And:}
\begin{displaymath}
\begin{array}{|c|c|c|}
A & \land & B  \\
\hline
T & T & T  \\
T & F & F  \\
F & F & T  \\
F & F & F  \\
\end{array}
\end{displaymath}

\textbf{Disjunction/Or:}
\begin{displaymath}
\begin{array}{|c|c|c|}
A & \lor & B  \\
\hline
T & T & T  \\
T & T & F  \\
F & T & T  \\
F & F & F  \\
\end{array}
\end{displaymath}

\textbf{Implication:}
\begin{displaymath}
\begin{array}{|c|c|c|}
A & \Longrightarrow & B \\
\hline
T & T & T  \\
T & F & F  \\
F & T & T  \\
F & T & F  \\
\end{array}
\end{displaymath}

\textbf{If and only if (Iff):}  
\begin{displaymath}
\begin{array}{|c|c|c|}
A & \Longleftrightarrow & B \\
\hline
T & T & T  \\
T & F & F  \\
F & F & T  \\
F & T & F  \\
\end{array}
\end{displaymath}

In the above, $A$ and $B$ are statements which can be either true (T) or false (F).  The result of the logical operation is shown in the column under the particular symbol.  For example, if A is true, and B is true, then $A \land B$ is true.  However, either either $A$ or $B$ is false (or both) then $A \land B$ is false.  It is important to refer back to the truth table definitions of these logical operations to avoid confusion.\\  

Sometimes, implication is called ``if-then" as in ``If A then B."  However, note that it is possible for A to be false and B to be true, and the implication $A \Longrightarrow B$ to be true.  This is somewhat at odds with how this if-then might be expressed in common usage: How could something false imply something true?  This seems confusing.  Therefore, always refer to the truth table for clarity on this point.\\

Here is an illustration regarding implication ($\Longrightarrow$).

Let $A=(a > 5), B=(a > 3)$ where $a$ is some integer.  Now consider the following cases: \\
If $a=6, (6 > 5)=\textit{T}, (6 > 3)=\textit{T}$ \\  
If $a=4, (4 > 5)=\textit{F}, (4 > 3)=\textit{T}$  \\
If $a=2, (2 > 5)=\textit{F}, (2 > 3)=\textit{F}$ \\

If we compare this to the truth table for implication, we see that implication is always true.  Note that with this particular definition of the statements $A$ and $B$ we cannot have a case where $A$ is true and $B$ is false. \\ 

\textit{Exercise}
Show $A  \Longleftrightarrow  B$ is equivalent to $\left(A \Longrightarrow B \right) \land \left(B \Longrightarrow A \right).$ 
\\
\\
\textit{Solution} \\
\begin{displaymath}
\begin{array}{|c|c|c|c|c|c|c|}
(A & \Longrightarrow & B)  & \land & (B & \Longrightarrow & A ) \\
\hline
T & T & T & T & T & T & T  \\
T & F & F & F & F & T & T  \\
F & T & T & F & T & F & F  \\
F & T & F & T & F & T & F  \\
\end{array}
\end{displaymath}

We now consider some important terminology:

\begin{itemize}
\item \textit{Tautology}:  Any statement that is true for all values of its components is a tautology.

\item \textit{Contradiction}:  Any statement that is false for all values of its components is a contradiction.  The negation of a contradiction is a tautology and the negation of a tautology is a contradiction.

\item \textit{Logically Equivalent}:  When $ A \Longleftrightarrow B $ is a tautology.  This means $A$ and $B$ have the same truth table.  The notation for this is $A \equiv B$.

\end{itemize}

We illustrate logical equivalence by considering ``De Morgan's Laws" \index{De Morgan's Law's} for disjunction (or) and conjunction (and).  De Morgan's Law's are useful for manipulating logical expressions: \\

$\neg\left(A \lor B \right) \equiv \left(\neg A\right) \land \left(\neg B \right)$

$\neg\left(A \land B \right) \equiv \left(\neg A\right) \lor \left(\neg B \right)$ \\

In words, the first law says that the negation $(\neg)$ of a disjunction ($A \lor B)$ is logically equivalent to the conjunction of the negations of the individual statements $A$ and $B$.  The second law is similar, except we exchange conjunction for disjunction and vice-versa.\\

\textit{Exercise}
Verify the De Morgan law for disjunction by showing the logical equivalence of the left and right hand sides using a truth table:\\
\textit{Solution}\\


\begin{displaymath}
\begin{array}{| c | c | c | c | c | c | c | c | c | c |}
\neg & (A & \lor  & B) & \Longleftrightarrow & (\neg & A) & \land & (\neg & B) \\
\hline
F & T & T & T & T & F & T & F & F & T \\
F & T & T & F & T & F & T & F & T & F \\
F & F & T & T & T & T & F & F & F & T \\
T & F & F & F & T & T & F & T & T & F \\
\end{array}
\end{displaymath}

We note that for any value of statements $A$ and $B$, the values under the column for $\Longleftrightarrow$ are all true, indicating that both expressions $\neg\left(A \lor B \right)$, $\left(\neg A\right) \land \left(\neg B \right)$ have the same truth tables, as required.\\

Another useful logical equivalence is the following ``contrapositive'' form of implication, which we explore in the next exercise:

\textit{Exercise}
Show that $\left(A \Longrightarrow B \right) \Longleftrightarrow \left(\neg B \Longrightarrow \neg A\right)$ is a tautology, i.e. this means that $A \Longrightarrow B $ is logically equivalent (LE) to $\neg B \Longrightarrow \neg A$.

\textit{Solution}\\

\begin{displaymath}
\begin{array}{| c | c | c | c | c | c | c | c | c |}
(A & \Longrightarrow & B)& \Longleftrightarrow & (\neg & B & \Longrightarrow & \neg & A) \\
\hline
T & T & T & T & F & T & T & F &T \\
T & F & F & T & T & F & F & F & T \\
F & T & T & T & F & T & T & T & F \\
F & T & F & T & T & F & T & T & F \\
\end{array}
\end{displaymath}


To show a situation where expressions are not logicaly equivalent, the following exercise may be considered:

\textit{Exercise} 
Show $A \Longrightarrow B$ is not logically equivalent to $B \Longrightarrow A$

\textit{Solution}\\

\begin{displaymath}
\begin{array}{| c | c | c | c | c | c | c |}
(A & \Longrightarrow & B) & \Longleftrightarrow & (B & \Longrightarrow & A) \\
\hline
T & T & T & T & T & T & T \\
T & F & F & F & F & T & T \\
F & T & T & F & T & F & F \\
F & T & F & T & F & T & F \\
\end{array}
\end{displaymath}

Notice the middle column $\Longleftrightarrow$ has false entries, so we don't have logical equivalency.\\

Another item of terminology is \textit{Logical Implication}.  Logical implication is when the expression $A \Longrightarrow B$ is a tautology.  If the statements $A$ and $B$ are such that we only get true statements from $A \Longrightarrow B$, then $A$ logically implies \textbf{(LI)} $B$.  This was the situation with the earlier illustration where we had defined statements $A$ and $B$ such that $A=(a > 5), B=(a > 3)$ where $a$ is some integer.

If $A$ is logically equivalent (LE) to $B$, then $A$ LI $B$ and $B$ LI $A$.

\textit{Exercise}:  Show $A\land \left(A \Longrightarrow B\right)$ logically implies $B$.  This is called ``Modus Ponens.''  

\textit{Solution}\\

\begin{displaymath}
\begin{array}{| c | c | c | c | c | c | c |}
 (A & \land & (A & \Longrightarrow & B)) & \Longrightarrow & B \\
\hline
T & T & T & T & T & T & T \\
T & F & T & F & F & T & F \\
F & F & F & T & T & T & T \\
F & F & F & T & F & T & F \\
\end{array}
\end{displaymath}


\section{Zermelo-Fraenkel Axiomatic Set Theory}

Set theory is at the foundation of mathematics.  The reader should have some awareness of the axioms of set theory, although in practice it seems that one rarely sees a mathematical proof that explicitly cites these axioms.  These axioms are referred to as the Zermelo-Frankel Axioms.

\subsection{The Zermelo-Fraenkel Axioms}

The following so-called Zermelo-Fraenkel Axioms of set theory are presented to help provide the reader a more comprehensive picture of mathematics foundations.  This section is inspired by the set of YouTube lectures by Richard Borcherds \href{https://www.youtube.com/playlist?list=PL8yHsr3EFj52EKVgPi-p50fRP2_SbG2oi}{(ZF Axioms)}\\
\verb+(https://www.youtube.com/playlist?list=PL8yHsr3EFj52EKVgPi-p50fRP2_SbG2oi)+\\ and also the book \textit{Axiomatic Set Theory} by Paul Bernays.  The definitions are adapted from Bernays.\\

We note that in the axioms below, the notion of \textit{set} is undefined, as is the inclusion relation, $\in$.  Furthermore, all elements or members of sets are sets themselves.\\

\textit{Axiom of Extensionality}

If $s \subseteq t$ and $t \subseteq s$, then $s=t$ where $s,t$ are sets and $\subseteq$ means \textit{subset}.\\  

We define subset by the following:  If $s$ and $t$ are sets such that, for all $x$, $x \in s$ implies $x \in t$, $s$ is called a subset of $t$, where $x$ is a set element or member.  Occasionally, we might use the notation $s \subset t$ to indicate a \textit{proper subset} whereby there is an element in $t$ that is not in $s$. \\

To recap:  The axiom states what it means for two sets to be equal - namely they contain the same elements.\\

\textit{Axiom of Foundation}

Every non-empty set $s$ contains an element $t$ such that $s$ and $t$ have no common element.\\

This axiom prevents an element of a set having an element which appears in the original set.  For example, the set $s=\{s\}$ is not allowed.\\

\textit{Axiom of Pairing}

For any two different sets $a$ and $b$, the pair $\{a,b\}$, or $\{b,a\}$, exists.\\

\textit{Axiom of Union}

For any set $s$ which contains at least two elements, there exists the set whose elements are the elements of the elements of $s.$ \\ 

This set is called the union of the elements of $s$ and is denoted by $\cup s.$  \\

\textit{Axiom of Infinity}

There exists at least one set $W$ with the properties:

\begin{itemize}
\item a) $\emptyset \in W$, where $\emptyset$ refers to the ``empty set." - the set with no elements.
\item b) if $x \in W$, then $\{x\} \in W.$
\end{itemize}

The interpretation of the above is that a set exists with a non-ending sequence of elements.  For example, if we think of the infinite sequence of non-negative integers being encoded as follows:  $0 = \emptyset, 1=\{ 0 \}, 2=\{1\}, \ldots$, then we can have the corresponding infinite set $S=\{\emptyset, \{\emptyset\}, \{\{\emptyset\}\}, \ldots \}$ which is justified by the axiom.\\

\textit{Axiom of Power-set}

For any set $s$, there exists the set whose elements are all subsets of $s$.\\

To take an example, consider the set $s = \{1,2,3 \}$.  Then the power-set is $P=\{\{1\}, \{2\}, \{3\}, \{1,2\}, \{1,3\}, \{2,3\}, \{1,2,3\}, \emptyset \}$.  The number of elements, also called the \textit{cardinality}\index{Cardinality}, in this case is given by $2^{N}$, where $N$ is the number of elements in the original set $s$.  Thus, we have $2^{N=3}=8$.  We note that the empty set, $\emptyset$, is considered a subset of every set.\\

\textit{Axiom of Subsets}  (Note: Borcherds refers to this as ``separation")

For any set $s$ and any predicate $P$ which is meaningful for all elements of $s$, there exists the set $y$ that contains just those elements $x$ of $s$ which satisfy the predicate $P$. \\

For example, let $s=\{1,2,3,4,5\}$ and the predicate $P\left(x\right)$ be true whenever $x$ is even.  Then, we have the set $y = \{x \in s | P (x) \textit{true} \} = \{2,4\}$.\\

\textit{Axiom of Replacement}

For every set $s$ and every single-valued function $f$ of one argument which is defined for all the elements of $s$, there exists the set that contains all $f(x)$ with $x \in s$.\\

\textit{Axiom of Choice}

For every disjointed set $t$ for which $\emptyset \notin t$, the Cartesian product $\mathfrak{P}t$ differs from the $\emptyset.$\\

We define a \textit{disjointed set} by the following:  If a set $s$ contains at least two elements, and any two elements of $s$ are mutually exclusive (i.e. these elements have no common elements themselves).\\  

\textbf{Theorem} The \textit{Cartesian product} of the elements of a disjointed set $t$, denoted $\mathfrak{P}t$ is the set whose elements are the sets which contain a single element from each element of $t$.  If $a,b, \ldots$ are the elements of $t$, the Cartesian product is also denoted by $a \times b \times \ldots$.  \textit{Note}: For those interested, this theorem is proved in Bernays. \\\\


To illustrate the applications of the above axioms to a proof, consider the following exercise.

\textit{Exercise}  Complete the proof of the following theorem (adapted from Bernays):\\

\textbf{Theorem} For every non-empty set $t$, there exists the set whose elements are common to all elements of $t$.\\
This set is called the \textit{intersection} of the elements of $t$ and is denoted by $\cap t.$\\

\textbf{Proof:}  By the Axiom of Union, $s=\cup t$ exists and contains the elements of the elements of $t$.  The predicate ``x is contained in each element of t'' defines, by the Axiom of \textbf{\underline{fill in the blank}}, a subset of $s$ which is the intersection $\cap t$.\\


\textit{Solution}\\
Axiom of \textbf{Subsets}\\\\

The following exercise explores the Cartesian product.

\textit{Exercise}  According to Bernays, for a disjointed set $t$, one has the Cartesian product, $\mathfrak{P}t=\emptyset$ if $\emptyset \in t.$  Prove this using the Cartesian Product Theorem stated above in Axiom of Choice section.\\

\textit{Solution}
According to the theorem, the Cartesian product is the set whose elements are the sets which contain a single element from each element of $t$.  Since $\emptyset \in t$ and contains no elements, the Cartesian product must be empty since there is no set possible that contains the required one element from $\emptyset$.\\

\subsection{Set Operations}
In this section, we present some set operations that are useful in much of the remainder of this book.  The reader may be familiar with some of the material, although perhaps not necessarily in the formalism presented. Concepts and proofs in mathematics are very frequently expressed in terms of set language to be precise, so utility with set operations is important.  For example, in later sections, we will discuss algebraic structures such as ``groups'' which will be defined in terms of sets with certain properties.  

The emphasis here will be to continue the development following the axiomatic approach, largely adapted from \textit{Axiomatic Set Theory} by Patrick Suppes.  Hopefully, this will illustrate some of the notation and concepts we have been developing in the previous sections.\\

\subsubsection{Binary Relations}

Definition:  $R \textit{ is a binary relation } \Longleftrightarrow (\forall x) (x \in R \Longrightarrow (\exists y)(\exists z)(x=\langle y,z \rangle))$.

We note that in the above, $\langle y, z \rangle$ is called an \textit{ordered pair} and is formally defined in set notation as $\langle y, z \rangle = \{\{y\}, \{y,z\}\}$.  Note how the ordering is encoded.  So we have that a relation is a set of ordered pairs.\\

The definition is read as ``$R$ is a binary relation if and only if for all $x$ if $x$ is in $R$, then there exists a $y$ and there exists a $z$ such that $x$ is equal to the ordered pair $\langle y, z \rangle$." \\

It should be remembered that the phrases ``if and only if" and the ``if ... then" are used to refer to the truth tables for the corresponding symbols and not the casual english meaning.  If desired, we could express the definition by using a truth table as we did previously to be completely clear.\\

Sometimes the following notation is used discussing relations: $x R y$.  This is formally defined as $x R y \Longleftrightarrow \langle x,y \rangle \in R$.\\

Consider the following example set which is a relation:

\begin{displaymath}
R=\{\langle 1,2 \rangle, \langle 2,1 \rangle,  \langle 1,3 \rangle,  \langle 3,1 \rangle,  \langle 2,3 \rangle,  \langle 3,2 \rangle \}.
\end{displaymath}

We have utilized the angle bracket notation for ordered pairs for simplicity.  Observe that this relation, $R$, has some special properties.  To begin with, this relation is called \textit{symmetric} because if $x R y$ is true then we have $y R x$ true.  For example, $\langle 1, 2 \rangle \in R $ and $\langle 2, 1 \rangle \in R$.\\

\textit{Exercise} For the relation $R$ defined above, verify that it is symmetric by explicitly listing the ordered pairs in question and their corresponding symmetric partners.\\

\textit{Solution}\\
$\langle 1, 2 \rangle, \langle 2, 1 \rangle$  \\
$\langle 2, 1 \rangle, \langle 1, 2 \rangle$  \\
$\langle 1, 3 \rangle, \langle 3, 1 \rangle$  \\
$\langle 3, 1 \rangle, \langle 1, 3 \rangle$  \\
$\langle 2, 3 \rangle, \langle 3, 2 \rangle$  \\
$\langle 3, 2 \rangle, \langle 2, 3 \rangle$  \\ 

We consider as an exercise, the proof that $\emptyset$ is a relation.\\

\textit{Exercise} Prove that $\emptyset$ (the empty set) is a relation by appealing to the definition of relation given earlier.\\
\textit{Solution}\\
Recalling the definition: $R \textit{ is a binary relation } \Longleftrightarrow (\forall x) (x \in R \Longrightarrow (\exists y)(\exists z)(x=\langle y,z \rangle))$, we substitute $\emptyset$ for $R$ and obtain:\\

$\emptyset \textit{ is a binary relation } \Longleftrightarrow (\forall x) (x \in \emptyset \Longrightarrow (\exists y)(\exists z)(x=\langle y,z \rangle))$.\\

Now, since $x \in \emptyset$ is false (the empty set contains no members), the statement $(x \in \emptyset \Longrightarrow (\exists y)(\exists z)(x=\langle y,z \rangle))$ is true.  Recall the truth table for $\Longrightarrow$ whereby if the first term is false, the entire statement is true no matter what the second term is.  Therefore, we have shown the $\emptyset$ is a relation because it satisfies the definition.\\

\subsubsection{Ordering Relations}

In the previous section, we introduced the idea of a \textit{symmetric} relation.  We now present a more complete list of such definitions:

\begin{itemize}
\item $R \textit{ is reflexive in }\index{reflexive} A \Longleftrightarrow (\forall x)(x \in A \Longrightarrow x R x)$.
\item $R \textit{ is irreflexive in }\index{irreflexive} A \Longleftrightarrow (\forall x)(x \in A \Longrightarrow \neg (x R x))$.
\item $R \textit{ is symmetric in }\index{symmetric} A \Longleftrightarrow (\forall x)(\forall y)(x,y \in A  \land x R y \Longrightarrow  y R x)$.
\item $R \textit{ is asymmetric in }\index{asymmetric} A \Longleftrightarrow (\forall x)(\forall y)(x,y \in A \land x R y \Longrightarrow \neg (y R x))$.
\item $R \textit{ is antisymmetric in }\index{antisymmetric} A \Longleftrightarrow (\forall x)(\forall y)(x,y \in A \land x R y \land y R x \Longrightarrow x = y)$.
\item $R \textit{ is transitive in }\index{transitive}A \Longleftrightarrow (\forall x)(\forall y)(\forall z)(x,y,z \in A \land x R y \land y R z \Longrightarrow x R z)$.
\item $R \textit{ is connected in }\index{connected} A \Longleftrightarrow (\forall x)(\forall y)(x,y \in A \land x \ne y \Longrightarrow x R y \lor y  R x)$.
\item $R \textit{ is strongly connected in }\index{strongly connected} A \Longleftrightarrow (\forall x)(\forall y)(x,y \in A \Longrightarrow x R y \lor y R x)$.

\end{itemize}

A couple of items regarding notation.  The notation $x,y \in A$ means $x \in A \land y \in A$.  Also, by convention, the $\land$ and $\lor$ operations are performed before $\Longrightarrow$ which saves writing extra parentheses in the definitions.  For example, if we explicitly included the parentheses in the definition of symmetric to indicate order of operations, it would appear as $A(\forall x)(\forall y)(\textbf{(}x,y \in A  \land x R y \textbf{)} \Longrightarrow  y R x)$.

If we consider the definition of \textit{symmetric} above, we see how this corresponds to the example symmetric  relation in the previous section.  The reader should be cautioned to carefully observe the difference between \textit{\underline{a}symmetric} and \textit{\underline{anti}symmetric} relations.\\

\textit{Exercise} Prove that the relation from the previous section is irreflexive by appealing to the definition.  Define the set $A=\{1,2,3\}$ for purposes of this exercise.\\

\textit{Solution}\\

Recalling $R$:

\begin{displaymath}
R=\{\langle 1,2 \rangle, \langle 2,1 \rangle,  \langle 1,3 \rangle,  \langle 3,1 \rangle,  \langle 2,3 \rangle,  \langle 3,2 \rangle \}.
\end{displaymath}

we see that $\forall x \in A$, namely $1,2,3$, we are missing the corresponding ordered pairs $\langle 1,1 \rangle, \langle 2,2 \rangle, \langle 3,3 \rangle$ in the relation $R$.  Now if we consider the definition of \textit{irreflexive} we have: $R \textit{ is irreflexive in } A \Longleftrightarrow (\forall x)(x \in A \Longrightarrow \neg (x R x))$.  Clearly, $x R x$ is false for any choice of $x \in A$, thus we have have $\neg (x R x)$ true for all choices of $x \in A$ and we see that $R$ is therefore indeed \textit{irreflexive}.\\

\textit{Exercise} Show by counterexample, that the relation from the previous section is not \textit{transitive}.  Once again, consider the set $A=\{1,2,3\}$ for purposes of this exercise.\\

\textit{Solution}\\

Recalling the definition: $R \textit{ is transitive in } A \Longleftrightarrow (\forall x)(\forall y)(\forall z)(x,y,z \in A \land x R y \land y R z \Longrightarrow x R z)$, we see that if we can find an instance where $x,y,z \in A \land x R y \land y R z \Longrightarrow x R z$ is false, we are done.  Consider $1 R 2$ and $2 R 1$ which are both true.  However, $1 R 1$ is false because $\langle 1,1 \rangle \notin R.$  Therefore, we conclude that $R$ is not transitive.\\

The following two definitions are important:

\begin{itemize}
\item $R \textit{ is a partial ordering of }\index{partial ordering}A \Longleftrightarrow R \textit{ is reflexive, anti-symmetric and transitive in }A$.
\item $R \textit{ well-orders }\index{well-order}A \Longleftrightarrow R \textit{ is connected in } A \land (\forall B)(B \subseteq A \land B \ne \emptyset \Longrightarrow B \textit{ has an }R \textit{-minimal element})$.
\end{itemize}

The definition of $R$\textit{-minimal element}\index{R-minimal element} mentioned above is as follows:
$x \textit{ is an }R\textit{-minimal element of }A \Longleftrightarrow x \in A \land (\forall y)(y \in A \Longrightarrow \neg(y R x))$.\\

Let us examine the above two ordering definitions with some examples.  Firstly, let us consider \textit{partial ordering}.  We will take $A=\{1,2,3\}$ and construct an appropriate relation.  We can begin with taking all possible ordered pairs which gives us the following provisional relation:

\begin{displaymath}
R=\{\langle 1,1 \rangle, \langle 1,2 \rangle, \langle 1,3 \rangle, 
\langle 2,1 \rangle,\langle 2,2 \rangle,\langle 2,3 \rangle, 
\langle 3,1 \rangle,\langle 3,2 \rangle,\langle 3,3 \rangle\}
\end{displaymath}

The above relation is symmetric, because it contains $\langle 1,1 \rangle, \langle 2,2 \rangle, \langle 3,3 \rangle$.  The anti-symmetric requirement restricts the ordered pair partners.  For example, if we have $\langle 1,2 \rangle$ we cannot have $\langle 2, 1\rangle$.  So, let us remove those ordered pairs where the second element is smaller than the first:

\begin{displaymath}
R=\{\langle 1,1 \rangle, \langle 1,2 \rangle, \langle 1,3 \rangle, 
\langle 2,2 \rangle,\langle 2,3 \rangle, \langle 3,3 \rangle\}
\end{displaymath}

Lastly, we need to verify transitivity:\\
$1R1 \land 1R1 \Longrightarrow 1R1$ \\
$1R1 \land 1R2 \Longrightarrow 1R2$ \\
$1R1 \land 1R3 \Longrightarrow 1R3$ \\
$1R2 \land 2R2 \Longrightarrow 1R2$ \\
$1R2 \land 2R3 \Longrightarrow 1R3$ \\
$1R3 \land 3R3 \Longrightarrow 1R3$ \\
$2R2 \land 2R2 \Longrightarrow 2R2$ \\
$2R2 \land 2R3 \Longrightarrow 2R3$ \\
$2R3 \land 3R3 \Longrightarrow 2R3$ \\
$3R3 \land 3R3 \Longrightarrow 3R3$ \\

So, our revised $R$ is a partial ordering of the set $A$.  When dealing with real numbers, the symbol $\le$ (less than or equal to) represents a partial ordering relation.  The relation $R$ we constructed above parallels this.\\

Now, let us do a similar construction to obtain a well-ordering.  Again, consider the set $A=\{1,2,3\}$ and let us construct a relation $R$ that satisfies the definition.  To begin with, we can take as provisional all possible ordered pairs:

\begin{displaymath}
R=\{\langle 1,1 \rangle, \langle 1,2 \rangle, \langle 1,3 \rangle, 
\langle 2,1 \rangle,\langle 2,2 \rangle,\langle 2,3 \rangle, 
\langle 3,1 \rangle,\langle 3,2 \rangle,\langle 3,3 \rangle\}
\end{displaymath}

Now, we first need to make sure that $R$ is \textit{connected} in $A$ by appealing to the definition of connected: $R \textit{ is connected in }\index{connected} A \Longleftrightarrow (\forall x)(\forall y)(x,y \in A \land x \ne y \Longrightarrow x R y \lor y  R x)$.  We have as members of $R$, $\langle 1,2 \rangle, \langle 1,3 \rangle, \langle 2,3 \rangle$, so $R$ is connected.\\

Next, we need to verify the second part of the definition: $(\forall B)(B \subseteq A \land B \ne \emptyset \Longrightarrow B \textit{ has an }R \textit{-minimal element})$.  There are seven subsets of $A$ (we don't need to consider $\emptyset$), which are listed here:\\
$\{1\}$\\
$\{2\}$\\
$\{3\}$\\
$\{1,2\}$\\
$\{1,3\}$\\
$\{2,3\}$\\
$\{1,2,3\}$\\

For $1$ to be an $R\textit{-minimal}$ in $B$, we must remove $\langle 1,1 \rangle$ from $R$.  Similar, arguments apply to $\langle 2,2 \rangle$ and $\langle 3,3 \rangle$ corresponding to $\{2\}$ and $\{3\}$.  Now, for $\{1,2\}$ we must remove $\langle 2,1 \rangle $ from $R$ so that $1$ is $R\textit{-minimal}$ in $\{1,2\}$, and likewise for $\{1,3\}$ we remove $\langle 3,1 \rangle $.  Similarly, we remove $\langle 3,2 \rangle$ corresponding to $\{2,3\}$.  Now, $\{1,2,3\}$ has an $R\textit{-minimal}$ element already, so we are done.  Our revised relation $R$ is now:

\begin{displaymath}
R=\{ \langle 1,2 \rangle, \langle 1,3 \rangle, \langle 2,3 \rangle\}
\end{displaymath}

This revised relation $R$ well-orders $A$.  This parallels the $<$ (less than) relation in real numbers.\\

\textit{Exercise} There is a theorem in Suppes (Theorem 62) as follows:  $R \textit{ well-orders } A \Longrightarrow R \textit{ is asymmetric and transitive in } A.$  Verify the relation $R$ above satisfies this theorem.\\

\textit{Solution}.  We first verify that $R$ is asymmetric.  We have $\langle 1,2 \rangle$ but not $\langle 2,1 \rangle$, $\langle 1,3 \rangle$ but not $\langle 3,1 \rangle$, and lastly $\langle 2,3 \rangle$ but not $\langle 3,2 \rangle$.  Now, we verify transitivity:  The only statement that makes sense is $1R2 \land 2R3 \Longrightarrow 1R3$ which is true.  Therefore, we have verified that $R$ satisfies the theorem.

\subsubsection{Equivalence Relations}

We now discuss some important concepts that will appear in some form later, notably in abstract algebra.  The reader will benefit by getting comfortable with these topics as early as possible, so we present them here.

\begin{itemize}
\item Definition:  $R \textit{ is an equivalence relation } \Longleftrightarrow R \textit{ is a relation } \land R \textit{ is reflexive, symmetric, and transitive}$.
\item Definition: the $R \textit{-coset of }\index{Coset} x$ is defined by $R[x]=\{y | x R y\}$.
\item Definition: $\Pi \textit{ is a partition of }\index{Partition} A \Longleftrightarrow \cup \Pi = A \land (\forall B)(\forall C)(B \in \Pi \land C \in \Pi \land B \ne C \Longrightarrow B \cap C = \emptyset) \land (\forall x)(x \in \Pi \Longrightarrow (\exists y)(y \in x))$.
\end{itemize}

A \textit{partition} of a set is an exhaustive, mutually exclusive decomposition of a set into subsets, whose union is the set itself.  For example, if our set is $A=\{1,2,3\}$, then a partition might be $\Pi = \{\{1\},\{2\}, \{3\} \}$ or perhaps $\Pi = \{\{1\},\{2, 3\} \}$.\\

We can see the partitioning through an example.  Let us consider a set $A=\{1,2,3,4,5\}$ and construct an equivalence relation, $R$.  Firstly, we know that $R$ must be reflexive, so we provisionally begin with:

\begin{displaymath}
R=\{ \langle 1,1 \rangle, \langle 2,2 \rangle, \langle 3,3 \rangle\, \langle 4,4 \rangle\, \langle 5,5 \rangle\}
\end{displaymath}

We could actually stop here, because $R$ happens to also be symmetric and transitive.  However, to make the example slightly more complex, let us add some more members that are symmetric:

\begin{displaymath}
R=\{ \langle 1,1 \rangle,  \langle 1,2 \rangle, \langle 2,1 \rangle, \langle 2,2 \rangle, \langle 3,3 \rangle\, \langle 4,4 \rangle\, \langle 5,5 \rangle\}
\end{displaymath}

where is can be seen that $\langle 1,2 \rangle$ and $\langle 2,1 \rangle$ have been added.\\
Now we need to verify that transitivity is satisified:\\
$1R1 \land 1R1 \Longrightarrow 1R1$ \\
$1R1 \land 1R2 \Longrightarrow 1R2$ \\
$1R2 \land 2R1 \Longrightarrow 1R1$ \\
$1R2 \land 2R2 \Longrightarrow 1R2$ \\
$2R1 \land 1R1 \Longrightarrow 2R1$ \\
$2R1 \land 1R2 \Longrightarrow 2R2$ \\
$2R2 \land 2R1 \Longrightarrow 2R1$ \\
$2R2 \land 2R2 \Longrightarrow 2R2$ \\
$3R3 \land 3R3 \Longrightarrow 3R3$ \\
$4R4 \land 4R4 \Longrightarrow 4R4$ \\
$5R5 \land 5R5 \Longrightarrow 5R5$ \\


Now, let us determine the $R \textit{-coset of }x$ for each $x \in A$:\\
$R[1]=\{1,2\}$\\
$R[2]=\{1,2\}$\\
$R[3]=\{3\}$\\
$R[4]=\{4\}$\\
$R[5]=\{5\}$\\

Now, we can form a set of all the cosets, which will be a partition of $A$:\\
$\Pi=\{\{1,2\},\{3\},\{4\},\{5\}\}$.

\subsubsection{Functions}

To help motivate the discussion of \textit{functions}, which presumably many readers will have encountered previously and have a fair degree of experience with, within the set theoretical framework, the following directly quoted passage from Suppes (p.86) may be helpful:\\

``Since the eighteenth century, clarification and generalization of the concept of a function have attracted much attention.  Fourier's representation of `arbitrary' functions (actually piecewise continuous ones) by trigonometric series encountered much opposition; and later when Weierstrass and Riemann gave examples of continuous functions without derivatives, mathematicians refused to consider them seriously.  Even today many textbooks of the differential and integral calculus do not give a mathematically satisfactory definition of functions.  An exact and completely general definition is immediate within our set-theoretical framework.  A function is simply a many-one relation, that is, a relation which to any element in its domain relates exactly one element in its range."\\

We now present the formal definition:\\

Definition: $f \textit{ is a function}\index{function} \Longleftrightarrow f \textit{ is a relation } \land (\forall x)(\forall y)(\forall z)(x f y \land x f z \Longrightarrow y = z)$.\\

Here is an example from Suppes which we use as an exercise.\\

\textit{Exercise} Argue that the following relation is not a function: $f=\{\langle 1,1\rangle, \langle 1,2 \rangle, \langle 3,4 \rangle\}$.\\

\textit{Solution}:\\
Appealing to the definition of function, we see that the statement $1f1 \land 1f2 \Longrightarrow 1=2$ is false, therefore we conclude $f$ is not a function.  In other words, the ordered pairs $\langle 1, 1 \rangle$ and $\langle 1, 2\rangle$ which appear in $f$ have the same first element, $1$, but different second elements, $1$ and $2$, which is not allowed for a function.\\

\textit{Composition} \index{composition} of functions is an important concept that appears quite frequently in some of the later topics.  The notation for this is $(f \circ g)(x)=f(g(x))$, where $f$ and $g$ are functions, $x$ is an element of a set, and the circle indicates composition.  In words, we can think of this as a chain, whereby, firstly the mapping is applied using the the $g$ function, then the result of this is mapped via the $f$ function.  Suppes states a formal definition of composition using something called the \textit{relative product}\index{relative product}, so for completeness we provide the definitions correspondingly:\\

Definition: $f \circ g = g/f$.\\

The relative product is given by the notation $g/f$ and is defined by the following:\\

Definition: $g/f = \{\langle x,y \rangle\| (\exists z)(x g z \land z f y )\}$.\\

\textit{Exercise}  Given the two functions $f=\{\langle 3,2 \rangle, \langle 4,5 \rangle, \langle 6, 7 \rangle \}$ and $g=\{\langle 1,3 \rangle , \langle 2,6 \rangle \}$, determine the composition $f \circ g$ using the definition $g/f$.

\textit{Solution}:\\
Let us pair up the terms $x g z$ and $z f y$ (in that order), where $z$ is common to both:\\
$\langle 1, 3 \rangle , \langle 3, 2 \rangle$\\
$\langle 2, 6 \rangle , \langle 6, 7 \rangle$\\

Then we have $g/f = \{\langle 1,2 \rangle, \langle 2, 7 \rangle \}$.\\

\textit{Exercise} Prove the following theorem (which is actually part of ``Theorem 82'' in Suppes): $f \textit{ and }g \textit{ are functions} \Longrightarrow f \circ g \textit{ is a function }$.\\

\textit{Solution}:\\  From the definition of a function provided earlier, we need to establish that $h=f \circ g$ is a relation, and also prove that $(\forall x)(\forall y)(\forall z)(x h y \land x h z \Longrightarrow y = z)$.  For the first part, we appeal to the definition $f \circ g= g/f$ and subsequently note that the definition of $g/f$ is a set of ordered pairs, which is a (binary) relation by definition.\\
For the second part, we need to show: $(\forall x)(\forall y)(\forall z)(x h y \land x h z \Longrightarrow y = z)$.  Now consider $x h y$.  From the definition of relative product, this means that $(\exists w)(x g w \land w f y)$, and likewise for $x h z$ this means that  $(\exists w')(x g w' \land w' f z)$. Rewriting the hypothesis, we obtain: $(\forall x)(\forall y)(\forall z)((\exists w)(x g w \land w f y) \land (\exists w')(x g w' \land w' f z))$.  Now, since $g$ is a function, this implies that $w=w'$.  Then, since $f$ is a function and $w=w'$, this implies $y=z$ thereby completing the proof.\\ 

We now define the concept of a converse relation, denoted $\stackrel{\smile}{R}$, which will be immediately useful in what comes next.\\

Definition: $\stackrel{\smile}{R}=\{\langle x,y \rangle | y R x\}$, where $R$ is a relation.\\

The following definitions are key:
\begin{itemize}
\item Definition: $f \textit{ is 1-1 } \Longleftrightarrow f \textit{ and } \stackrel{\smile}{f} \textit{ are functions }$.
\item Definition: $f \textit{ is 1-1 } \Longrightarrow f^{-1}= \stackrel{\smile}{f}$, where $f^{-1}$ is the \textit{inverse}\index{inverse} of $f$.
\end{itemize}

We note that ``1-1" is read ``one-one."


\section{Peano Axioms of Arithmetic}

\section{Godel Theorems}