\chapter{Mathematical Foundations}

The objective of this chapter is to present some basic concepts that underlie much of the remaining material in this book.

\section{Basic Notation}

The reader may be familiar with some if not all of the following notation, but it is presented here for concreteness and review.

\begin{itemize}
\item $\forall$ is read as ``for all"  
\item $\exists$ is read as ``exists"    
\item $\subset$ is read as ``subset of"  
\item $\in$ is read as ``in"  
\item $\mathbb{Z}$ is the set of integers: $\ldots, -2,-1,0,1,2, 3, \ldots$
\item $\mathbb{N}$ is the set of natural numbers: $0,1,2, 3, \ldots$ (note: Sometimes these start with 0, and sometimes with 1.  Usually doesn't matter to the argument in question, but the reader should be aware.)
\item $\mathbb{Q}$ is the set of rational numbers.  These are numbers of the form $\frac{p}{q}$ where $p,q$ are integers, and $q \ne 0$
\item $\mathbb{R}$ is the set of real numbers.  These will later be defined precisely in terms of \textit{Dedikind Cuts},\index{Dedekind Cuts} but for now we'll just use the informal notion of all the points on a coordinate line.  However, hopefully the reader can see this is not good enough for precise work.
\item We use curly braces to indicate sets in general.  For example, $\mathbb{Z}=\left\{\ldots, -2,-1,0,1,2,3 \ldots\right\}$ defines the set of integers using the curly braces and examples.  It is hoped the reader is able to abstract that the dots to the left imply the list has no smallest negative value and the dots to the right imply that there is no largest positive value.  
\item Set builder notation:\index{Set builder notation}  There are other ways to specify the set within the braces.  For example, we can indicate the property that each member of the set has:  $S = \left\{p | p=2m+1, m \in \mathbb{Z} \right\}.$  This is read as the set $S$ consists of those values $p$ of the form $2m+1$ where $m$ is an integer.  In other words, $S$ the set of odd integers.  The vertical bar in the braces is read as ``such that" and the comma is read as ``and.''
\end{itemize}

\textit {Exercise}

Write the definition of the set of rational numbers using set builder notation.

\textit{Solution}

$\mathbb{Q}=\left\{r | r = \frac{p}{q}, p,q \in \mathbb{Z}, q \ne 0 \right\}$

\section{Mathematical Logic}

\section{Zermelo Frankel Axiomatic Set Theory} 

\section{Peano Axioms of Arithmetic}

\section{Godel Theorems}